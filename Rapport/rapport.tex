% Document description
\documentclass[a4paper,11pt]{report}
\usepackage[utf8]{inputenc}
\usepackage[T1]{fontenc}
\usepackage{lmodern}
\usepackage[francais]{babel}
\usepackage{listings}
\usepackage{graphicx} %Pour inclure les images
\usepackage{float} %Pour plus de précision sur le placement
\usepackage{color}
\usepackage[hidelinks]{hyperref} %Pour les liens dans le PDF
\usepackage{fancyhdr} %En-tête + pieds de page
\usepackage{lastpage}

% Metadata
\title{HMIN317 - Moteur de jeux \\ Compte-rendu TP1}
\author{BOYER Benoît}
\date{Septembre 2017}

\input{css.tex}


% --------------> Document beginning <--------------
\begin{document}

    \section{Question 1}
    \subsection{A quoi servent les classes MainWidget et GeometryEngine ?}
    La classe MainWidget sert à gérer l'affichage et les évènements du programme, en effet si on se fie aux foncions présentes dans le .h :
    \lstinputlisting[language=C++, caption=Fonctions de la classe MainWidget, firstline=75, lastline=85]{../mainwidget.h}
    Les fonctions dans l'ordre permettent :
    \begin{itemize}
    	\item De gérer le clic souris
    	\item De gérer le relachement du clic souris
    	\item De gérer le temps (et faire un mouvement continu, comme par exemple attrapper le cube et lui inculquer un mouvement pour qu'il continue et ralentisse sur le temps)
    	\item D'initialiser la fenêtre OpenGL
    	\item De gérer la modification de la taille de la fenêtre
    	\item De redessiner et mettre à jour le contenu de la fenêtre OpenGL
    	\item D'initialiser les shaders
    	\item D'initialiser les textures (donc les charger et de les préparer pour les utiliser plus tard)
    \end{itemize}   
    \hfill \break
	La classe GeometryEngine est utilisée pour la géométrie du dé présent :
    \lstinputlisting[language=C++, caption=Fonctions de la classe GeometryEngine, firstline=64, lastline=67]{../geometryengine.h}
    Les fonctions présentes servent à dessiner et à initialiser le cube qui, une fois texturé, sera le dé à la fenêtre.

    \pagebreak
    \subsection{A quoi servent les fichiers fshader.glsl et vshader.glsl ?}
    Les fichiers sont dans l'ordre le {\it{fragment shader}} et le {\it{vertex shader}}. Le fragment shader va servir à appliquer la bonne couleur au pixel (en se basant sur la texture fournie), tandis que le vertex shader va calculer la position des vertices par rapport à la fenêtre.
	
	\pagebreak
	\section{Question 2}
	\subsection{Expliquer le fonctionnement des deux fonctions de la classe CubeGeometry.}
	    La fonction {\lstinline{void initCubeGeometry()}} permet d'initialiser le cube en créant dans un premier temps les vertices du cube :
    \lstinputlisting[language=C++, firstline=85, lastline=99, caption=Création de deux faces du cube]{../geometryengine.cpp}
    Pour ensuite dans un second temps, recenser les indices pour faire les triangles des faces des cubes :
    \lstinputlisting[language=C++, firstline=126, lastline=140, caption=Création des indices]{../geometryengine.cpp}
    Et pour terminer, on transfère les données au GPU via des buffers :
    \lstinputlisting[language=C++, firstline=143, lastline=149, caption=Transfert des données via un VBO]{../geometryengine.cpp}
    Une fois terminé, la fonction {\lstinline{void drawCubeGeometry(QOpenGLShaderProgram *program)}} doit être utilisée pour dessiner le cube.\\ \\*
    Dans un premier temps, on va indiquer à OpenGL quels VBO ({\it{vertex shaders}}) utiliser :
    \lstinputlisting[language=C++, firstline=156, lastline=158, caption=Selection des VBO]{../geometryengine.cpp}
    Ensuite, on indique au pointeur le début du VBO, puis le lieu des données :
    \lstinputlisting[language=C++, firstline=160, lastline=166, caption=Indication aux vshader des données]{../geometryengine.cpp}
    \pagebreak
    
    Ensuite, on indique où se trouve les coordonnées pour les textures dans la mémoire, puis on la donne au buffer :
    \lstinputlisting[language=C++, firstline=168, lastline=174, caption=Indication des coordonnées de texture]{../geometryengine.cpp}
    
    Et pour finir, on dessine le cube.
    \lstinputlisting[language=C++, firstline=176, lastline=177, caption=Dessin du cube]{../geometryengine.cpp}
	
	\pagebreak
	\section{Question 3}
	\subsection{Création d'une surface plane}
    	
\end{document}